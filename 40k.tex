% !TEX encoding = UTF-8 Unicode
\documentclass[paper=landscape]{scrartcl}

% Not quite elegant, but it works
%
% Source:
% https://warhammer40k.fandom.com/wiki/Imperial_Dating_System


\usepackage{color}
\typearea{30}
\usepackage{multicol}

\input fantasycalender.code.tex

\FantasyCalendarAddMonths{6}{
  January=31, 
  February=28, 
  March=31, 
  April=30, 
  May=31, 
  June=30, 
  July=31 , 
  August=31, 
  September=30, 
  October=31 ,
  November=30 ,
  December=31 ,
}

\FantasyCalendarAddDays{Monday, Tuesday, Wednesday, Thursday, Friday, Saturday, Sunday}


\DefineLeapYear
  {
    year divisible by 4;
    but not divisible by 100;
    but divisible by 400;
  }
  { February=+1 }




\begin{document}

\ExplSyntaxOn

\cs_new:Npn \PrintTripleDigit #1
  {
      \ifnum#1<10\relax
        00\relax
      \else
        \ifnum#1<100\relax
          0
        \fi
      \fi
      \int_eval:n {#1}
  }

  \cs_new:Npn \GetCorrectYear #1
    {
      \int_set:Nn \l_tmpa_int { \int_div_truncate:nn {#1} { 1000 } }
      \int_set:Nn \l_tmpb_int { #1 - \l_tmpa_int*1000 }
      \PrintTripleDigit \l_tmpb_int
      .
      M \int_eval:n { \l_tmpa_int + 1 }
    }
  \cs_new:Npn \GetCorrectFraction #1 #2
    {
      \int_set:Nn \l_tmpa_int 
        { 
          \fp_eval:n { floor ( ( (#1) * 24 + #2) *0.11407955 )  }
        }
      \PrintTripleDigit \l_tmpa_int
    }
  

\tikzset
  {
    day ~ text=
      {
        \%d0 \\
        0
        .
        \GetCorrectFraction { \%s0 - 1 } { 0 }
        .
        \GetCorrectYear{ \%y0 }
      }
  }
\ExplSyntaxOff

\centering
\PrintFantasyCalendarYear[box height=2.9cm, box width=3.6cm]{\the\year}
  {
%    
  }

\PrintFantasyCalendarYear[box height=2.9cm, box width=3.6cm]{40000}
  {
%    
  }

  
\end{document}