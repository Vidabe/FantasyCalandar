% !TEX encoding = UTF-8 Unicode
\documentclass[paper=landscape]{scrartcl}

\usepackage{color}
\typearea{30}

\input fantasycalender.code.tex

%\FantasyCalendarAddMonths{2}
%  {
%    A = 24,
%    B = 36,
%    C = 47 ,
%    D = 58,
%    E = 69,
%    F = 80,
%    G = 91
%  }
%
%\FantasyCalendarAddDays{ Q, W, E, R, T, Z }

\FantasyCalendarAddMonths{2}{
  Basoth=40, 
  Eganar=45, 
  Dumena=40, 
  Sapara=40, 
  Paraduk=45, 
  Siduk=45, 
  Mushera=40 , 
  Eskian=40, 
  Zianna=45, 
  Ribrun=40 ,
}

\FantasyCalendarAddDays{Agguan, Sielan, Flethan, Wanuran, Usyn, Jundal, Izach, Verhan, Haturan}


%
\FantasyCalendarDateEntry{1-1}{Neujahr}
\FantasyCalendarDateEntry{2-40}{Erntedankfest}
\FantasyCalendarOneTimeEntry{0-2-40}{Noping}


\begin{document}
\ExplSyntaxOn


\newcount\cnta
%\newcount\theyear
%\newcount\themonth
\newcount\cntday
\__fantasycalendar_iso_to_ordinal:nN { 0-1-1 } \cnta

\the \cnta

\__fantasycalendar_ordinal_to_date:nNNN { \cnta } \theyear \themonth \theday

\par
\the \cnta
\space\theyear \space \themonth \space \theday
\par
\__fantasycalendar_ordinal_to_weekday:nN { \cnta } \cntday
\the\cntday
\__fantasycalendar_print_day_name:n{\cntday}

\par

\basefantasycalendar{stuff}{0-1-1}{0-last-last}{ \ifdate {start ~ of ~ month} {\par} {\space}\l_fantasycalendar_current_day_tl  }


\par
\newpage
\tikzpicture\PrintFantasycalendarBase[months=0-1 to 0-1, week~list, box ~  size=2cm,days={rectangle,draw},month ~ label ~ above ~ centered];\endtikzpicture

\newpage

\DefineNewMoon\Solena { 419 } { 0 }
\NewFantasyMoon \Surios { 21 } { 0 }
\NewFantasyMoon \Eria { 42 } { 0 } %% 23

%\printmoon{}{}{}
A
\Solena { 0 }
\Solena { 8 }
\Solena { 10 }
\Solena { 5 }
\Solena { 4 }
\Solena { 12 }
\Solena { 15 }
\Solena { 1 }
\Solena { 99 }
\par

\GetNextMoonphase \Solena { new-moon } { iso=0-1-1 } 
  {  
     \%{ordinal}\space In: \%{in-days}\ days. Iso-Date: \%{iso} \par
  }

\GetNextMoonphase \Solena { new-moon } { iso=0-1-1 } 
  {  
     \%{ordinal}\space In: \%{in-days}\ days. Iso-Date: \%{iso} \par
  }

\GetNextMoonphase \Surios {new-moon} { iso=0-1-1 } 
  {       
    \%{ordinal}\space In: \%{in-days}\ days. Iso-Date: \%{iso} \par
  }

\GetNextMoonphase  \Eria {new-moon} { iso=0-1-1 } 
  {  
     \%{ordinal}\space In: \%{in-days}\ days. Iso-Date: \%{iso} \par
  }

\par

\CompareMoons { \Solena, \Eria } { 0 } { iso=0-1-1, nr-of-entries=10 } 
  {
    \%{ordinal}\space In: \%{in-days}\ days. \%{iso} \par
 }




\Solena { 9009 }
\Eria { 9009 }


\end{document}